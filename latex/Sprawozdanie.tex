

\title{Laboratorium 3 - Sprawozdanie}
\author{Wojciech Makuch}

\maketitle
\section{Zadanie}
Program framework benchmarkujacy dla zaimplentowanej struktury danych Stos.
\section{Realizacja}
Program zawiera 3 struktury danych. Każda z nich zawiera 3 podstawowe metody: połóż element, zdejmij element, zwróć rozmiar. Struktua Stos zbudowana za pomocą tablicy z realokacją pamięci, Lista ze wskaźnikiem na następny element, oraz Kolejka z indeksami na pierwszy i ostatni element. Wszystkie struktury danych działają prawidłowo.
Ponadto program zawiera funkcje wypęłniającą struktury liczbami psełdolosowymi oraz zliczającą czas dla przeprowadzenia testów złożonosci obliczeniowej ww. struktur.
\section{Działanie}
Głowna funkcja programu działa tylko na strukturze typu Stos. Testuje jego 2 metody push() oraz push200(). Metoda push() polega na powiekszaniu alokowanej pamieci o 1 element, natomiast push200() polega na alokowaniu pamieci razy 200\%. Działanie programu polega na zliczeniu czasu wypelniania tej struktury liczbami pseudolosowymi oraz zapisania wyników do pliku o nazwie \textsl{Pomiar\_czasu2.txt.}
\section{Wyniki}
Podczas alokowania pamieci struktura typu stos obsługiwana przez metode push() może alokować pamięć na maksimum $10^{5}$ elementów. Natomiast dzięki metodzie push2000() rozmiar ten zostaje zwiększony do $10^{7}$. Ponadto można zauważyć dłuższy czas destrukcji struktury zaalokowanej przez push().Z danych dostarczonych przez program wynika, że metoda push200() działa o wiele szybciej i jest w stanie zaalokować więcej pamięci. Na Rys 1. pokazano w skali logarytmicznej wykres zależności ilości elementów od czasu potrzebnego na wypełnienie nimi struktury.
\begin{figure}[h!]
\centering
\includegraphics[scale=0.7]{wykres1}
\caption{Wykres złożoności obliczeniowej}
\label{fig:wykres1}
\end{figure}
Z wykresu można zauważyć, że złożoność obliczeniowa jest w przybliżeniu liniowa, czyli O(n). Ponadto złożoność metody push() rośnie o wiele szybciej, co znaczy, że program ma gorszą złożonoś obliczeniową. Dodatkowo na wykresie zaznaczono dla porównania przebieg idealnej funkcji liniowej. Widać, że metoda push200() ma nawet lepszą złożoność niż O(n). :)
\section{Komentarz}
Do utworzenia dokumentacji wykorzystano system Doxygen.
Funkcja pomiaru czasu dla systemu Windows pobrana ze strony dr. J. Mierzwy. Program skompilowano w środowisku Code::Blocks. Do stworzonia wykresu posłużono się pakietem MS Excel, sprawozdanie napisano używając systemu \LaTeX.
\end{document}